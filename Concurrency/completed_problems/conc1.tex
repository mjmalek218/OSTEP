\documentclass{article}

\usepackage{amsmath}
\usepackage{amsfonts}
\usepackage[margin = .5in]{geometry}

\begin{document}

\noindent \textbf{Homework for Intro to Concurrency}  %% FILL THIS IN


\begin{enumerate}

\item 
 
\textbf{Instruction Run}: ./x86.py -p loop.s -t 1 -i 100 -R dx

loop.s should subtract 1 from \%dx, check to see if the result is negative, and if it isnt, 
repeat the subtraction until it is.  

Once I ran the ``-c'' command it cleared up some confusion I had regarding the way the output
was formatted: it traces the register but prints it out *next to* a single instruction, not
after an entire run through of the code. This is more verbose than I expected but better
I guess for detailed inspection. 

So since there was only run through, it makes perfect sense that initial value was 0.

\item 

\textbf{Instruction Run}: ./x86.py -p loop.s -t 2 -i 100 -a dx=3,dx=3 -R dx

Upon first seeing this I immediatley got hung up on the double initialization of the same
register. But what I finally realized is that in a multi-threaded program, *every single 
thread has its own set of registers*. The shared portion of the thread is main memory, not 
registers or program counter, both of which preserved via context switches. 

Also note the interrupt every one hundred instructions is essentially meaningless here, as 
it takes far less than a hundred instructions to terminate. 

So to finally answer, the question, I predicted duplicate behavior by the threads, repeating
the subtraction operation until \%dx goes from 3 to -1...and that is exactly what happened.

\item 

Instruction run: ./x86.py -p loop.s -t 2 -i 100 -a dx=3,dx=3 -R dx

No, it doesn't. And why would it? We're operating on pre-initialized registers that are 
unique to each thread, *not* to shared memory locations which are the crux of the issue. All
that happens is the threads interleaved: otherwise the behavior is preserved. 

\item 

\textbf{Instruction Run}: ./x86.py -p looping-race-nolock.s -t 1 -M 2000

Looking at the 'cat' terminal output, the code moves the memory from location 2000 into register
\%ax, adds 1 to it, then moves it back. After this is done the code subtracts 1 from \%bx, tests
to see if \%bx is 0 yet. If it isnt, it jumps back up to the top. 

Even without running the program it is is clear this code is inefficient. Rather than merely
keeping the value inside the register and checking the condition there, it moves the value
back to memory upon every iteration before checking the condition. 

I predicted that the value would be 1 after the run, since we hadn't explicitly set it with
program arguments and the program only loops once. After checking this answer was confirmed.

\item  

\textbf{Instruction Run}: ./x86.py -p looping-race-nolock.s -t 2 -a bx=3 -M 2000

Since each register is set separately by itself, i.e. each thread has its own set of registers,
setting bx=3 will set the registers in *separate* threads to 3. Since bx is also the loop counter,
and the loop decrements one at a time, both threads loop three times then terminate. 

The final result will be double the loop coutner since memory is shared between threads (unlike
resiters), or 6. 

\item 

\textbf{Instruction Run}: ./x86.py -p looping-race-nolock.s -t 2 -M 2000 -i 4 -r -s 0

The critical section is when the assembly code has moved the memory value into the register. 
The problem with having multiple threads access this is that they may both move the memory
value into the register to be updated with no way of coordinating this updating, alter it in
their own ways, and then move it back. This could lead to multiple issues...

Note that the critical section is conveniently denoted in the comments given to us in the 
code...

\item 

\textbf{Instruction Run}: ./x86.py -p looping-race-nolock.s -a bx=1 -t 2 -M 2000 -i 1

Since the thread is interrupted every single instruction, it seems to me as though this
should be equivalent to a single thread running tbh. In other words, in effect only
one add operation will have been performed to x (so the final x value will be 1) whereas
two might have been intended. 

and...running the c option confirms my hunch. I would guess that once the interval is set to take
up the entire critical section in one go, or i = 3, then the final memory value will indeed be 2.
G

and...again my hunch is confirmed. The final result with $i = 3$ is that $x = 2$. 

\item Basically, in order to get a correct result the interrupt frequency needs to be a multiple
of 3, or the size of the critical section. Otherwise with the large number of loops, a thread will
eventually be interrupted while in the critical section. Evidently, the *maximum* number will
always be if the code runs correctly, or 200, and this cannot be exceeded by the nature of the 
range condition. That is because the number of looping times is set in separate registers,
so all that may be overwritten are the increments. 

Of course, the multiple of three rule stops mattering once the interrupt level is set to be large
enough so the first thread is allowed to terminate before it can be interrupted. 

\item 

\textbf{Instruction Run}: ./x86.py -p wait-for-me.s -a ax=1,ax=0 -R ax -M 2000

From looking at the code, it seems as though depending upon the value of the register ax for the 
thread, the code is either a ``waiter'' (for ax = 0) or a ``signaller'' (for ax = 1). 
If it's a waiter thread, it moves the memory value located in address 2000 into register cx,
tests it against 1, and if its not equal to that, repeats. 

The signaller essentially is the only way this operation can be terminated...once the signaller
is run it finally moves the desired ``1'' value for the waiter into the 2000 memory address
and the waiter can finish. 

With the given parameters, this won't even be a problem since the signaller is called ``before''
the waiter. Running the code we see that this is indeed the case and that eahc thread only
runs once.

Its final value will be 1: the value moved there by the signaller. 

\item 

\textbf{Instruction Run}: ./x86.py -p wait-for-me.s -a ax=0,ax=1 -R ax -M 2000

Fairly straightforward...thread 1 essentially ``spins'' or locks indefinitely until the signaller
is called and the thread is interrupted. I'm not sure what the default interrupt rate is but
as -i was not set in the instruction I surmise there exists one, and that was what allowed
the signaller, or thread 2, to be called, and for the wait loop to be ended.  

Obviously the program is not efficiently using the CPU since it's redundantly checking address
2000 when nothing has changed, and it does so many times in a loop. 

\end{enumerate}




\end{document} 