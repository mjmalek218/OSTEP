\documentclass{article}

\usepackage{amsmath}
\usepackage{amsfonts}
\usepackage[margin = .5in]{geometry}

\begin{document}

\noindent \textbf{Homework for Thread API}  %% FILL THIS IN


\begin{enumerate}

\item 
 
\textbf{Instruction Run}: valgrind --tool=helgrind ./main-race 

(By the way the instruction is run, it is clear that *valrind* is the name of a 
suite of test tools, and *helgrind* is a *sub-tool* within this suite). 

\item The race condition is obvious: balance++ can be incremented by either the 
worker or the main thread. 

Now as for helgrind: it seems to work by first running the program, with a series
of stdout declarations to the terminal marked as ``Thread Announcement.'' It does
this for each thread created, in the order they were created, and seems to list
some memory address next to each executed instruction (perhaps the location
of the instruction in memory...need to look into this). 

It subsequently reports two data races (which makes sense...as there are two
threads that make up the race) one by thread \#1 and one by thread \#2. For
each of these it gives a stack trace, followed by the conflicting data access
variable (balance). Note it even gives the size of the data subjected to the 
race condition: in this case, 4 bytes. So yes, it does give both offending lines
(8 and 15) in the code. 

\item 

\textbf{Instruction Run}: valgrind --tool=helgrind ./main-race 

Now after the above is run with the ``balance++'' commented out in the worker function, 
it doesn't even report the creation of the threads, which is interesting. So not only are 
the error messages removed, basically all of the output is removed as well. 

Next, we uncomment the offending line in the worker function, and instead add a lock
around both instances of balance. The result is the same as if we removed one of the 
offending lines: no error reported. 

\item 

No instruction was run here, just examined the code. The deadlock occurs because
of the ordering of the locking/unlocking of the mutex: the first created thread locks on
m1, the second created thread on m2, and the main thread waits for the first thread it created
to join. Every thread gets stuck and the program dead-locks. 


\item  

\textbf{Instruction Run:} valgrind --tool=helgrind ./main-deadlock

The first thing helgrind reports is the creation of the *last* thread under a ``thread accouncement'': this is  quite interesting to me, as this is the *last* thread created in the program. 
Perhaps there is a sort of ``stacking'' in the order of thread output? Anyways the call 
stack for thread #3 is given...

Yes I just realized this makes sense, as functions called all exist on a call stack.
In the case of threads of course these stacks are different, but I wonder whether or
not the calling thread actually retains information about the function called on its own 
stack? I'll have to examine this in further depth on Pintos. 

After the announcement of the creation of thread 3, it reports a violation of
the lock order. Looking at the code...thread 3 will lock on mutex 2, or &m2. 
Helgrind senses this deadlock and goes haywire..but I don't understand how to interpret
the output, as there are a significant # of hex memory references.  


\item 

\textbf{Instruction Run:} valgrind --tool=helgrind ./main-deadlock-global

The problem is different...both the child threads lock on a new mutex: the same
mutex g, unlike before where they locked on m1 and m2 , respectively. But
valgrind is reporting the exact same output...


\item 


\item 


\item 


\end{enumerate}




\end{document} 